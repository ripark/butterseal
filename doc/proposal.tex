\documentclass[letterpaper]{article}
\usepackage[margin=1cm]{geometry}
\usepackage{csquotes}

\pagenumbering{gobble}

\title{Game Development Proposal}
\author{%
  Sean Allred \and
  Natalie Cunningham \and
  Kyle Easterling \and
  Libby Glasgow \and
  Christina Wang}

\begin{document}
\maketitle

\section{Milestones}
\label{sec:milestones}

\begin{description}
\item[Milestone 1]
  By this milestone, we will have a working game engine.
  This at least includes having a firm control on
    the graphics,
    the audio,
    and the physical user interface
    of the game.
  As a proof of progress, we will be able to
    move about on the map as the player.
  We will acquire working music samples,
    and we will have working sprites for all major objects in the game to date.
\item[Milestone 2]
  By this milestone, the plot will be detailed enough to pass as a game.
  (What exactly this entails is of course subjective, but it's a fair assessment.)
  In addition, collision handling will be dealt with without error
    and the player will be able to interact with other things on the map.
  All sprites will be in for the game.
  If desired, backdrops will be produced for dialogue scenes and cut-scenes.
  Concept art will yet continue as we define the game.
  A method for animations (walking, running, flying) will be in-place, and
    all major puzzles will be conceptually defined.
\item[Milestone 3]
  By this final milestone, all assets for the game will be frozen.
  By this point, we expect to be ready to make our golden copy.
  We will make sure all documentation for the game is in order and deploy.
\end{description}

\section{Description}
\label{sec:description}

\def\GameTitle{\textit}

Our game is set in a fantasy world in a forest.
It is a puzzle-based game with an end goal of \enquote*{saving the day}---literally.
As the game progresses when you complete more puzzles,
  time also progresses and the scenery reflects the transition from day to night.

The driving aspect of the game is the music.
At the start, there is no music.
As you complete puzzles, you find parts of the music that serve as the background music.
The more parts you find, the more depth the music has until you finally collect all the parts and have a full song with all the instruments playing.
At this point, morning arrives, thus beating the game.

This game is heavily influenced by \enquote*{classic} 8-bit \textsc{rpg} games
  such as \GameTitle{The~Legend of~Zelda}
  (specifically \GameTitle{The~Ocarina of~Time}),
  \GameTitle{Fire~Emblem}, and
  \GameTitle{Pok\'emon}.
While the 8-bit style alone will evoke a certain nostalgia for (our) childhood,
  this game will be even \emph{more} fun from its sense of adventure,
  its challenging puzzles, and of course
  its unforgiving silliness.

\section{Flash Fiction}
\label{sec:flash-fiction}

\def\thought{\textsl}

It's early evening.
You wake up from your late-afternoon nap to
  a wet set of thick whiskers
  brushing across your face.

\enquote{The name's Butter,} the whiskers say.
Opening your eyes, you discover that the whiskers belong to a seal!
Startled, you sit \emph{straight} up on that once-comfortable couch
  to see just how clumsy a seal can be in a dinky cottage.
\enquote{Stay calm---I'm here to help you.}
\thought{Why is there a talking seal in my house!?}
\enquote{%
  There has been an accident, and
  if we can't fix this before sunrise tomorrow, well\dots
  sunrise won't come!}
\thought{Why is there a talking seal in my house!?}

\enquote{%
  You see, someone has stolen
  the voices of the forest
  not too far from here.
  Without their song, the birds won't sing!
  Do you know what this means!?}
\thought{Who let a talking seal in my house!?}
\enquote{%
  If I can't restore the
  voices of the forest,
  the birds won't sing,
  the sun won't rise,
  and the \emph{entire world} will be
  plunged into darkness \emph{forever}.
  We can't let that happen.}
\thought{Who let a talking seal in my house!?}

\enquote{%
  But I can't do it alone.
  As you can see, moving around is a little hard for me, \emph{especially} in a forest.
  A forest obviously isn't a large body of water, so that's out.
  I don't have legs, so I can't walk.
  The forest is too dense, so I can't very well fly.
  I need you.
  What do you say?}

\bigskip

\begin{center}
\thought{\dots Why is there a talking seal in my house!?}

\vfill

\thought{Wait\dots did that seal just say \enquote*{fly}?}
\end{center}

% \newpage

% so you know how butter goes away sometimes and comes back to help 
% well
% his evil twin seal-brother, margarine the walrus, leads him astray
% when the main character finally finds out, he says \enquote{I can't believe it's not butter!}

\end{document}

%%% Local Variables: 
%%% mode: latex
%%% TeX-master: t
%%% End: 
