\documentclass{article}
\usepackage{csquotes}

\title{Proposal}
\author{%
  Sean Allred \and
  Natalie Cunningham \and
  Kyle Easterling \and
  Libby Glasgow \and
  Christina Wang}

\begin{document}
\maketitle
\noindent

\section{Milestones}
\label{sec:milestones}

\subsection{Milestone 1}
\label{sec:milestone-1}
\begin{itemize}
\item Having a `working' basic engine
  (graphics, controls, sound).
\item We can move about in our grid.
\item We will have at least part of the map done
  and be able to walk around that area
\item Plot (and branches) draft complete.
\item Music working samples acquired.
\end{itemize}

\subsection{Milestone 2}
\label{sec:milestone-2}
\begin{itemize}
\item All puzzles conceptually designed.
\item Collision handling in the map.
\item interaction with other objects
\item All art assets in
\item Animations working in game engine
\end{itemize}

\subsection{Milestone 3}
\label{sec:milestone-3}
\begin{itemize}
\item Plot frozen.
\item Music frozen.
\item Art frozen.
\end{itemize}

\section{Influences}
\label{sec:influences}

\begin{itemize}
\item pok\'emon
\item fire emblem
\item oracle of seasons\slash ages
\item ocarina of time (music aspect)
\item Kirby (collecting dragon pieces)
\item old zelda games
\end{itemize}

\section{Description}
\label{sec:description}

Top-down \textsc{rpg}.
8-bit.

Our game is set in a fantasy world in a forest.
It is a puzzle-based game with an end goal of \enquote{saving the world.}
As the game progresses and you complete more puzzles,
  time also progresses and the scenery reflects the transition from day to night.
Once the player completes the game, morning arrives.
The driving aspect of the game is the music.
At the start, there is no music.
As you complete puzzles, you find parts of the music that serve as the background
music.
The more parts you find, the more depth the music has until you finally collect all the parts and have a full song with all the instruments playing.
At this point, morning arrives, thus beating the game.


\newpage

\section{Flash Fiction}
\label{sec:flash-fiction}

\def\thought{\textsl}

It's early evening.
You wake up from your late-afternoon nap to
  a wet set of thick whiskers
  brushing across your face.

\enquote{The name's Butter,} the whiskers say.
Opening your eyes, you discover that the whiskers belong to a seal!
Startled, you sit \emph{straight} up on that once-comfortable couch
  to see just how clumsy a seal can be in a dinky cottage.
\enquote{Stay calm---I'm here to help you.}
\thought{Why is there a talking seal in my house!?}
\enquote{%
  There has been an accident, and
  if we can't fix this before sunrise tomorrow, well\dots
  sunrise won't come!}
\thought{Why is there a talking seal in my house!?}

\enquote{%
  You see, someone has stolen
  the voices of the forest
  not too far from here.
  Without their song, the birds won't sing!
  Do you know what this means!?}
\thought{Who let a talking seal in my house!?}
\enquote{%
  If I can't restore the
  voices of the forest,
  the birds won't sing,
  the sun won't rise,
  and the \emph{entire world} will be
  plunged into darkness \emph{forever}.
  We can't let that happen.}
\thought{Who let a talking seal in my house!?}

\enquote{%
  But I can't do it alone.
  As you can see, moving around is a little hard for me, \emph{especially} in a forest.
  A forest obviously isn't a large body of water, so that's out.
  I don't have legs, so I can't walk.
  The forest is too dense, so I can't very well fly.
  I need you.
  What do you say?}

\thought{\dots Why is there a talking seal in my house!?}

\vfill

\thought{Wait\dots did that seal just say \enquote*{fly}?}

\vfill

\newpage

so you know how butter goes away sometimes and comes back to help 
well
he's evil twin seal-brother, margarine the walrus, leads him astray
when the main character finally finds out, he says \enquote{I can't believe it's not butter!}

\end{document}

%%% Local Variables: 
%%% mode: latex
%%% TeX-master: t
%%% End: 
